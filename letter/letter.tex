\documentclass[letterpaper,11pt]{letter}
\usepackage[T1]{fontenc}
\usepackage[utf8]{inputenc}
\usepackage{times}
\usepackage{lucimono}

\usepackage[unicode,colorlinks,citecolor=black,linkcolor=violet,urlcolor=blue,
pdftitle=Letter\ to\ American\ Society\ of\ Mammalogists,
pdfsubject=Natural\ History,
pdfauthor=Pipfrosch\ Press]{hyperref}

\address{Michael A. Peters\\Pipfrosch Press\\xxx Xxxxx Pl.\\Brentwood, CA 94513\\(925) xxx-xxxx}
\signature{Michael A. Peters}

\begin{document}

\begin{letter}{American Society of Mammalogists\\Address of the destination\\Phone number of the destination}

\opening{Dear ASM,}

% corps of the letter
The Center for Biological Diversity, The Alongside Wildlife Foundation, and the East Bay Regional Park District have been copied on this letter in hopes that they see the importance of this request and are able to offer me advice on how to obtain the permission I seek if this letter is not enough.

Pipfrosch Press is an organization I have started with a purpose of citizen science education and the promotion of biological diversity conservation awareness, particularly but not exclusively in Contra Costa County, California.

I am seeking permission to republish an article from the Journal of Mammalogy, Volume 22, Number 2 (14 May 1941) as an open access document in the digital ePub format. The specific article I wish to republish is `Natural History of the Tulare Kangaroo Rat' (pages 117--148) by Donald T.\ Tappe.

I believe that article contains crucial natural history information that is of substantial value to citizen science efforts to find a remnant population of a close relative, the Berkeley Kangaroo Rat (\textit{Dipodomys heermanni berkeleyensis}), last confirmed extant in 1940 and now presumed extinct by many.

Currently many in wildlife management take the position that habitat set aside for the federally protected Alameda Whipsnake (\textit{Coluber [=Masticophis] lateralis euryxanthus}) doubles as protection for any remnant populations of the Berkeley Kangaroo Rat that may persist, but I believe that position is fallacious.

Urbanization and cats are cited for the decline and possible extinction of the Berkeley Kangaroo Rat. That makes sense in the Berkeley and Oakland hills, but it does not explain their decline within the Diablo foothills or the Briones hills. My \emph{suspicion} is their decline in those areas is actually the result of both Wild Boar (\textit{Sus scrofa scrofa}) from undomesticated Russian stock intentionally released for sport hunting and feral domestic pigs (\textit{Sus scrofa domesticus}) that are also present within those hills. Kangaroo rats store food in their tunnels. \textit{Sus scrofa} has a legendary nose and may destroy those tunnels to get at it. It also may destroy much of the short grass habitat on the edge of chaparral that are used as food gathering grounds for the Kangaroo Rat. This is of course speculation but I do not buy that urbanization and domestic cats caused their decline within the open areas where neither are present. Lack of fires changing the plant fauna dynamics of the chaparral may also have played a role. I am sure I do not have to tell you this, but extinction \emph{frequently} is the result of multiple factors that individually could have been weathered by a population.

If a remnant population is found, wildlife management would be able to study the population dynamics and discover what conservation measures need to be taken that may not have been in put place for the preservation of the Alameda Whipsnake. First we need to find a population to study. If we can find a population, answers to the questions about what caused the decline will be easier to find and proper habitat preservation and restoration can be attempted.

The article by Donald T.\ Tappe on the natural history of their close relative, the Tulare Kangaroo Rat, is very well written and contains very detailed information that likely is applicable to the natural history of the Berkeley Kangaroo Rat. I can not ask hobbyists who enjoy hiking in those open areas to pay over \$30 for access to a PDF document that is difficult to read on a phone. Even those who can afford it would probably rather spend it on a couple field guides that cover numerous species they are likely to encounter. A free ePub document that does display well on their phone is much more likely to be read and read with passion once they experience the high quality that Donald T.\ Tappe put into that article.

The Journal of Mammalogy has a well-deserved reputation for quality. I do not expect permission to be granted without some assurance of a quality reproduction, and being new, Pipfrosch Press does not yet have the reputation for quality I hope to have in five years.

To that end, I have already created the ePub for that journal article so that you can see the very high quality I put into transcribing the journal article to the ePub format. It is available at \url{https://pipfrosch.com/tappe/} which is not a URL I have publicly shared.

If you view that ePub document, you should see that I did far more than just transcribe the text into the XHTML format used within ePub. I did a high quality job paying careful attention to content accessibility practices so that users with disabilities that have related access barriers are far more likely to be able to enjoy the content of the article without relying upon assistance from a third party. Autonomy is important.

With all due respect to JSTOR for the important service they do provide, the accessibility of their PDF documents are often downright horrid. I understand there is a lot of cost involved in optimizing accessibility, but they do charge a rather high access fee. The ePub has an appendix chapter explaining the various measures I took to make sure the ePub version of the article is as accessible as I could make it, and it compares this accessibility with the JSTOR PDF. Many people who do not experience accessibility barriers are unaware of how deficient accessibility often is, which is why I provided that appendix chapter.

There are some quality improvements that can still be made. The photographic plates are the same resolution as the images within the JSTOR PDF but they are still too low, especially when viewed on high resolution devices. eBay often has historic print versions of academic articles, I am watching it so I can possibly acquire Volume 22 there and photograph the color plates with a DSLR mounted on a copy stand. I actually did find Volume 1 through 9 that way, in reasonable condition at a very affordable price (cheaper for all nine volumes than a JSTOR PDF for a single article) but I have not yet seen Volume 22 for sale there. Another option, sometimes libraries will provide high quality scans of journals in their possession for a fee. I may pursue that option if I can not locate a copy of Volume 22 for myself. If by any chance the American Society of Mammalogists is able to provide high quality scans of Plate 1 and Plate 2 from that article, I would be incredibly grateful.

I am also not happy with my SVG recreation of Figure 1. I tried to put too much detail into the coast. While looking at the excellent range maps Joseph Grinnell used for his 1922 publication `A geographical study of the kangaroo rats of California' it became very apparent that the coast line is a case where less detail is better. I am also not happy with the rivers, I should have used Bezier curves to create them but I used a polyline data plot which unfortunately has the effect of de-emphasizing their curvature. Figure 1 will be recreated. It will take me awhile, but it will be recreated to a higher visual quality. In my admittedly conceited opinion, it is already a higher visual quality than the scan of Figure 1 within the JSTOR PDF. I want better. It gnaws at me when I know I can do better, I think that is part of my autism.

With respect to improved accessibility, I am still in the process of both identifying words that a screen reader may mispronounce and finding the correct IPA phoneme string to represent the correct pronunciations.

Also for accessibility, a glossary of terms needs to be provided. It is my hope that Pipfrosch Press will be able to work out a deal with a company that already maintains a set of vetted glossary terms and definitions used in a scientific context.

I would like to stress that republishing academic journal articles still under copyright protection is not something Pipfrosch Press will do with high frequency. This is a special case where the benefit to citizen science is extremely high, it may assist in the location of a species currently presumed extinct.

\closing{With thanks,}

\cc{Center for Biological Diversity\\P.O.\ Box 710\\Tucson, AZ 85702}
\cc{Center for Biological Diversity\\1212 Broadway, St.\ \#800\\Oakland, CA 94612}
\cc{Dr.\ David A.\ Steen\\The Alongside Wildlife Foundation\\P.O.\ Box 12422\\Gainesville, FL 32603}
\cc{East Bay Regional Park District\\2950 Peralta Oaks Court\\Oakland, CA 94605}
%\ps{PS: PostScriptum}
%\encl{Enclosures}

\end{letter}
\end{document}
